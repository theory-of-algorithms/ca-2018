\documentclass[a4paper, 12pt]{exam}

% Exam class is broken without this.
\makeatletter
\expandafter\providecommand\expandafter*\csname ver@framed.sty\endcsname
{2003/07/21 v0.8a Simulated by exam}
\makeatother

% Enables the use of colour.
\usepackage{xcolor}
% Syntax high-lighting for code. Requires Python's pygments.
\usepackage{minted}
% Enables the use of umlauts and other accents.
\usepackage[utf8]{inputenc}
% Diagrams.
\usepackage{tikz}
% Settings for captions, such as sideways captions.
\usepackage{caption}
% Symbols for units, like degrees and ohms.
\usepackage{gensymb}
% Latin modern fonts - better looking than the defaults.
\usepackage{lmodern}
% Allows for columns spanning multiple rows in tables.
\usepackage{multirow}
% Better looking tables, including nicer borders.
\usepackage{booktabs}
% More math symbols.
\usepackage{amssymb}
% More math layouts, equation arrays, etc.
\usepackage{amsmath}
% More math fonts, like mathbb.
\usepackage{amsfonts}
% More theorem environments.
\usepackage{amsthm}
% More column formats for tables.
\usepackage{array}
% Adjust the sizes of box environments.
\usepackage{adjustbox}
% Better looking single quotes in verbatim and minted environments.
\usepackage{upquote}
% URLs.
\usepackage[hidelinks]{hyperref}
% Better blank space decisions.
\usepackage{xspace}
% Better looking tikz trees.
\usepackage{forest}
% For plotting.
\usepackage{pgfplots}

% Various tikz libraries.
\usetikzlibrary{mindmap}
\usetikzlibrary{shadows}
\usetikzlibrary{arrows}
\usetikzlibrary{positioning}
\usetikzlibrary{chains}
\usetikzlibrary{fit}
\usetikzlibrary{shapes}
\usetikzlibrary{decorations.markings}
\usetikzlibrary{calc}
\usetikzlibrary{arrows.meta}

% GMIT colours.
\definecolor{gmitblue}{RGB}{20,134,225}
\definecolor{gmitred}{RGB}{220,20,60}
\definecolor{gmitgrey}{RGB}{67,67,67}

% Rename Bibliography to a smaller "Refereces".
\renewcommand{\refname}{\selectfont\normalsize References} 

% Stop minted high-lighting errors.
\makeatletter
\expandafter\def\csname PYGdefault@tok@err\endcsname{\def\PYGdefault@bc##1{{\strut ##1}}}
\makeatother

% Set the header and footer.
\pagestyle{headandfoot}
\header{\textbf{CA 2018}}{}{Theory of Algorithms}
\footer{}{Page \thepage\ of \numpages}{}

% Change some things to do with marks.
\marksnotpoints
\pointsinrightmargin

% Empty cover page.
\begin{coverpages}
\end{coverpages}

% Print answers
\printanswers

\begin{document}

\noindent
The following exercises should be completed in the Racket programming language~\cite{racketwebsite}.
Remember to plan your work and make regular commits to your repository.
The instructions for submitting your work are given on the Moodle page.
Note that ``from scratch'' means using only \mintinline{racket}{cons}, \mintinline{racket}{car}, \mintinline{racket}{cdr}, \mintinline{racket}{define}, \mintinline{racket}{lambda}, \mintinline{racket}{if}, \mintinline{racket}{null}, \mintinline{racket}{null?}, \mintinline{racket}{cond}, \mintinline{racket}{=} and the four basic numerical operators (\mintinline{racket}{+}, \mintinline{racket}{-}, \mintinline{racket}{*}, \mintinline{racket}{/}).
Other basic functions may be allowed, but please confirm their use with the lecturer.

\begin{questions}

\question
Write, from scratch, a function in Racket that uses a brute-force algorithm that takes a single positive integer and return true if the number is a prime and false otherwise. Call the function \mintinline{racket}{decide-prime}.


\question
Write, from scratch, a function in Racket that takes a positive integer $n_0$ as input and returns a list by recursively applying the following operation, starting with the input number.
$$
n_{i+1} =
\left\{
  \begin{array}{ll}
    3n_i + 1 & \textrm{if } n_i \textrm{ is odd}  \\
		n_i \div 2 & \textrm{otherwise} \\
	\end{array}
\right.
$$
End the recursion when (or if) the number becomes $1$.
Call the function \mintinline{racket}{collatz-list}.
So, \mintinline{racket}{collatz-list} should return a list whose first element is $n_0$, the second element is $n_1$, and so on.
For example:
\begin{minted}{racket}
> (collatz-list 5)
'(5 16 8 4 2 1)
> (collatz-list 9)
'(9 28 14 7 22 11 34 17 52 26 13 40 20 10 5 16 8 4 2 1)
> (collatz-list 2)
'(2 1)
\end{minted}


\question
Write, from scratch, two functions in Racket.
The first is called \mintinline{racket}{lcycle}.
It takes a list as input and returns the list cyclically shifted one place to the left.
The second is called \mintinline{racket}{rcycle}, and it shifts the list cyclically shifted one place to the right.

For example:
\begin{minted}{racket}
> (lcycle (list 1 2 3 4 5))
'(2 3 4 5 1)
> (rcycle (list 1 2 3 4 5))
'(5 1 2 3 4)
\end{minted}


\end{questions}

\bibliographystyle{plain}
\bibliography{bibliography}
\end{document}
